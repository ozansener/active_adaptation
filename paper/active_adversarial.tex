%%%%%%%%%%%%%%%%%%%%%%%%%%%%%%%%%%%%%%%%%%%%%%%%%%%%%%%%%%%%%%%%%%
%%%%%%%% ICML 2017 EXAMPLE LATEX SUBMISSION FILE %%%%%%%%%%%%%%%%%
%%%%%%%%%%%%%%%%%%%%%%%%%%%%%%%%%%%%%%%%%%%%%%%%%%%%%%%%%%%%%%%%%%

% Use the following line _only_ if you're still using LaTeX 2.09.
%\documentstyle[icml2017,epsf,natbib]{article}
% If you rely on Latex2e packages, like most moden people use this:
\documentclass{article}

% use Times
\usepackage{times}
% For figures
\usepackage{graphicx} % more modern
%\usepackage{epsfig} % less modern
\usepackage{subfigure} 

% For citations
\usepackage{natbib}

% For algorithms
\usepackage{algorithm}
\usepackage{algorithmic}

% As of 2011, we use the hyperref package to produce hyperlinks in the
% resulting PDF.  If this breaks your system, please commend out the
% following usepackage line and replace \usepackage{icml2017} with
% \usepackage[nohyperref]{icml2017} above.
\usepackage{hyperref}

% Packages hyperref and algorithmic misbehave sometimes.  We can fix
% this with the following command.
\newcommand{\theHalgorithm}{\arabic{algorithm}}

% Employ the following version of the ``usepackage'' statement for
% submitting the draft version of the paper for review.  This will set
% the note in the first column to ``Under review.  Do not distribute.''
%\usepackage{icml2017} 

% Employ this version of the ``usepackage'' statement after the paper has
% been accepted, when creating the final version.  This will set the
% note in the first column to ``Proceedings of the...''
\usepackage[accepted]{icml2017}


% The \icmltitle you define below is probably too long as a header.
% Therefore, a short form for the running title is supplied here:
\icmltitlerunning{Adversarial Approach to Active Learning}

\begin{document} 

\twocolumn[
\icmltitle{Adversarial Approach to Active Deep Learning}

% It is OKAY to include author information, even for blind
% submissions: the style file will automatically remove it for you
% unless you've provided the [accepted] option to the icml2017
% package.

% list of affiliations. the first argument should be a (short)
% identifier you will use later to specify author affiliations
% Academic affiliations should list Department, University, City, Region, Country
% Industry affiliations should list Company, City, Region, Country

% you can specify symbols, otherwise they are numbered in order
% ideally, you should not use this facility. affiliations will be numbered
% in order of appearance and this is the preferred way.

\begin{icmlauthorlist}
\icmlauthor{Ozan Sener}{to}
\icmlauthor{Silvio Savarese}{to}
\end{icmlauthorlist}

\icmlaffiliation{to}{Department of Computer Science, Stanford University, CA, US}

\icmlcorrespondingauthor{Ozan Sener}{ozan@cs.stanford.edu}

% You may provide any keywords that you 
% find helpful for describing your paper; these are used to populate 
% the "keywords" metadata in the PDF but will not be shown in the document
\icmlkeywords{boring formatting information, machine learning, ICML}

\vskip 0.3in
]

% this must go after the closing bracket ] following \twocolumn[ ...

% This command actually creates the footnote in the first column
% listing the affiliations and the copyright notice.
% The command takes one argument, which is text to display at the start of the footnote.
% The \icmlEqualContribution command is standard text for equal contribution.
% Remove it (just {}) if you do not need this facility.

%\printAffiliationsAndNotice{}  % leave blank if no need to mention equal contribution
\printAffiliationsAndNotice{} % otherwise use the standard text.
%\footnotetext{hi}

\begin{abstract} 
Abstract.
\end{abstract} 

\section{Introduction}
Intro
\section{Related Work}
Related
\section{Weakly Supervised Learning with Adversery}
\subsection{Problem Definition - Active Learning with a Pool}
Simple notation
\subsection{Adversarial Learning as a Weak Supervision}
Just give probability ratio and explain why it makes to use it as a scale
\section{Active Learning as a Space Covering}
Just explain we will first understand theory then will explain the method inspired by it
\subsection{Analysis of Test Error for Active Learning}
\subsubsection{Robustness and Generalization}
Summary of classical results
\subsubsection{Extension of Classical Results to Active Deep Learning}
\begin{itemize}
\item Theorem 1. Every robust algorithm is robust with subset with closeness approximation
\item Proposition 1. A network with k layer of Conv+MaxPool+ReLu and d layer of FC is robust
\end{itemize}
\subsection{Active Learning Method}
Key idea: we need labelled points to cover the space so i) make distributions closer and ii) choose active points such that $\gamma$ is minimized
\section{Summary of the Algorithm with Implementation Details}
Give the algorithm in algorithmic format with full details
\section{Experiments}
Compare with state-of-the-art weak supervision and active learning
\section{Electronic Submission}
\label{submission}

\subsection{Algorithms}

If you are using \LaTeX, please use the ``algorithm'' and ``algorithmic'' 
environments to format pseudocode. These require 
the corresponding stylefiles, algorithm.sty and 
algorithmic.sty, which are supplied with this package. 
Algorithm~\ref{alg:example} shows an example. 

\begin{algorithm}[tb]
   \caption{Bubble Sort}
   \label{alg:example}
\begin{algorithmic}
   \STATE {\bfseries Input:} data $x_i$, size $m$
   \REPEAT
   \STATE Initialize $noChange = true$.
   \FOR{$i=1$ {\bfseries to} $m-1$}
   \IF{$x_i > x_{i+1}$} 
   \STATE Swap $x_i$ and $x_{i+1}$
   \STATE $noChange = false$
   \ENDIF
   \ENDFOR
   \UNTIL{$noChange$ is $true$}
\end{algorithmic}
\end{algorithm}
 
\subsection{Tables} 
% Note use of \abovespace and \belowspace to get reasonable spacing 
% above and below tabular lines. 

\begin{table}[t]
\caption{Classification accuracies for naive Bayes and flexible 
Bayes on various data sets.}
\label{sample-table}
\vskip 0.15in
\begin{center}
\begin{small}
\begin{sc}
\begin{tabular}{lcccr}
\hline
\abovespace\belowspace
Data set & Naive & Flexible & Better? \\
\hline
\abovespace
Breast    & 95.9$\pm$ 0.2& 96.7$\pm$ 0.2& $\surd$ \\
Cleveland & 83.3$\pm$ 0.6& 80.0$\pm$ 0.6& $\times$\\
Glass2    & 61.9$\pm$ 1.4& 83.8$\pm$ 0.7& $\surd$ \\
Credit    & 74.8$\pm$ 0.5& 78.3$\pm$ 0.6&         \\
Horse     & 73.3$\pm$ 0.9& 69.7$\pm$ 1.0& $\times$\\
Meta      & 67.1$\pm$ 0.6& 76.5$\pm$ 0.5& $\surd$ \\
Pima      & 75.1$\pm$ 0.6& 73.9$\pm$ 0.5&         \\
\belowspace
Vehicle   & 44.9$\pm$ 0.6& 61.5$\pm$ 0.4& $\surd$ \\
\hline
\end{tabular}
\end{sc}
\end{small}
\end{center}
\vskip -0.1in
\end{table}
 
\subsection{Citations and References} 

Please use APA reference format regardless of your formatter
or word processor. If you rely on the \LaTeX\/ bibliographic 
facility, use {\tt natbib.sty} and {\tt icml2017.bst} 
included in the style-file package to obtain this format.

Citations within the text should include the authors' last names and
year. If the authors' names are included in the sentence, place only
the year in parentheses, for example when referencing Arthur Samuel's
pioneering work \yrcite{Samuel59}. Otherwise place the entire
reference in parentheses with the authors and year separated by a
comma \cite{Samuel59}. List multiple references separated by
semicolons \cite{kearns89,Samuel59,mitchell80}. Use the `et~al.'
construct only for citations with three or more authors or after
listing all authors to a publication in an earlier reference \cite{MachineLearningI}.

Use an unnumbered first-level section heading for the references, and 
use a hanging indent style, with the first line of the reference flush
against the left margin and subsequent lines indented by 10 points. 
The references at the end of this document give examples for journal
articles \cite{Samuel59}, conference publications \cite{langley00}, book chapters \cite{Newell81}, books \cite{DudaHart2nd}, edited volumes \cite{MachineLearningI}, 
technical reports \cite{mitchell80}, and dissertations \cite{kearns89}. 


\bibliography{active_adversarial}
\bibliographystyle{icml2017}

\end{document} 


% This document was modified from the file originally made available by
% Pat Langley and Andrea Danyluk for ICML-2K. This version was
% created by Lise Getoor and Tobias Scheffer, it was slightly modified  
% from the 2010 version by Thorsten Joachims & Johannes Fuernkranz, 
% slightly modified from the 2009 version by Kiri Wagstaff and 
% Sam Roweis's 2008 version, which is slightly modified from 
% Prasad Tadepalli's 2007 version which is a lightly 
% changed version of the previous year's version by Andrew Moore, 
% which was in turn edited from those of Kristian Kersting and 
% Codrina Lauth. Alex Smola contributed to the algorithmic style files.  
