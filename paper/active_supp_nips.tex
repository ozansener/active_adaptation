\documentclass{article}

% if you need to pass options to natbib, use, e.g.:
% \PassOptionsToPackage{numbers, compress}{natbib}
% before loading nips_2017
%
% to avoid loading the natbib package, add option nonatbib:
% \usepackage[nonatbib]{nips_2017}

\usepackage{nips_2017}

% to compile a camera-ready version, add the [final] option, e.g.:
% \usepackage[final]{nips_2017}

\usepackage[utf8]{inputenc} % allow utf-8 input
\usepackage[T1]{fontenc}    % use 8-bit T1 fonts
\usepackage{hyperref}       % hyperlinks
\usepackage{url}            % simple URL typesetting
\usepackage{booktabs}       % professional-quality tables
\usepackage{nicefrac}       % compact symbols for 1/2, etc.
\usepackage{microtype}      % microtypography

\usepackage{graphicx} % more modern
\usepackage{caption}
\usepackage{subcaption}
\usepackage{wrapfig}

% AMS stuff
\usepackage{amsmath, amsthm, amssymb}
\usepackage{dsfont}
\newtheorem{theorem}{Theorem}
\newtheorem{lemma}{Lemma}
\newtheorem{cor}{Corollary}

% For algorithms
\usepackage{algorithm}
\usepackage{algorithmic}

\usepackage{booktabs}
%\newcommand{\theHalgorithm}{\arabic{algorithm}}

\DeclareMathOperator*{\argmin}{arg\,min}
\DeclareMathOperator*{\argmax}{arg\,max}
\usepackage{xspace}
\newcommand*{\eg}{e.g.\@\xspace}
\newcommand*{\ie}{i.e.\@\xspace}


\title{Active Learning for Convolutional Neural Networks\\ Supplementary Materials}

% The \author macro works with any number of authors. There are two
% commands used to separate the names and addresses of multiple
% authors: \And and \AND.
%
% Using \And between authors leaves it to LaTeX to determine where to
% break the lines. Using \AND forces a line break at that point. So,
% if LaTeX puts 3 of 4 authors names on the first line, and the last
% on the second line, try using \AND instead of \And before the third
% author name.

\author{
  David S.~Hippocampus\thanks{Use footnote for providing further
    information about author (webpage, alternative
    address)---\emph{not} for acknowledging funding agencies.} \\
  Department of Computer Science\\
  Cranberry-Lemon University\\
  Pittsburgh, PA 15213 \\
  \texttt{hippo@cs.cranberry-lemon.edu} \\
  %% examples of more authors
  %% \And
  %% Coauthor \\
  %% Affiliation \\
  %% Address \\
  %% \texttt{email} \\
  %% \AND
  %% Coauthor \\
  %% Affiliation \\
  %% Address \\
  %% \texttt{email} \\
  %% \And
  %% Coauthor \\
  %% Affiliation \\
  %% Address \\
  %% \texttt{email} \\
  %% \And
  %% Coauthor \\
  %% Affiliation \\
  %% Address \\
  %% \texttt{email} \\
}

\begin{document}
% \nipsfinalcopy is no longer used

\maketitle

\begin{abstract}
In this supplementary materials, we present the proofs which are skipped in the main text due to the space constraint.
\end{abstract}


\section{Proof for Lemma 1}
\begin{proof}
We will start with showing that softmax function defined over $C$ class is $\frac{\sqrt{C-1}}{C}$-Lipschitz continuous. It is easy to show that for any differentiable function \mbox{$f:\mathbb{R}^n\rightarrow\mathbb{R}^m$},

\[
\left \| f(\mathbf{x})-f(\mathbf{y})\right \|_2 \leq \left \|J\right \|^*_F  \left\| \mathbf{x}-\mathbf{y}\right\|_2  \, \, \forall \mathbf{x},\mathbf{y}\in\mathbb{R}^n
\]
where $\left \|J\right \|^*_F = \max\limits_{\mathbf{x}} \left \|J\right \|_F$ and $J$ is the Jacobian matrix of $f$.

Softmax function is defined as
\[
f(x)_i = \frac{\exp(x_i)}{\sum\limits_{j=1}^{C}\exp(x_j)}, \, i={1,2,...C}
\]
For brevity, we will denote $f_i(x)$ as $f_i$. The Jacobian matrix will be,
\[
J = \begin{bmatrix} f_1(1-f_1) & -f_1f_2  & ... & -f_1f_C \\
-f_2f_1 & f_2(1-f_2)  & ...  & -f_2f_C \\
... & ... & ... & ...  \\
-f_{C}f_{1} & -f_{C}f_{2}  & ...  & -f_{C}(1-f_{C})
\end{bmatrix}
\]
Now, Frobenius norm of above matrix will be,
\[
\left \| J \right \|_F = \sqrt{\sum\limits_{i=1}^{C}\sum\limits_{j=1 \\ i\neq j}^{C}f_{i}^{2}f_{j}^{2} + \sum\limits_{i=1}^{C} f_i^2(1-f_i)^2}
\]
It is straightforward to show that $f_i = \frac{1}{C}$ is the optimal solution for $\left \| J \right \|^{*}_F = \max\limits_{x}\left \| J \right \|_F $ Hence, putting $f_i = \frac{1}{C}$ in above equation , we get \mbox{$\left \| J \right \|^{*}_F = \frac{\sqrt{C-1}}{C}$}.

Now, consider two inputs $\mathbf{x}$ and $\mathbf{\tilde{x}}$, such that their representation at layer $d$ is $\mathbf{x}^d$ and $\mathbf{\tilde{x}}^d$. Let's consider any convolution or fully-connected layer as $\mathbf{x}^d_j = \sum_i w_{i,j}^d \mathbf{x}^{d-1}_i$. If we assume, \mbox{$\sum_i |w_{i,j}| \leq \alpha \quad \forall i,j,d$}.  For any convolutional or fully connected layer, we can state:
\[
\|\mathbf{x}^d - \mathbf{\tilde{x}}^d\|_2 \leq  \alpha \|\mathbf{x}^{d-1} - \mathbf{\tilde{x}}^{d-1}\|_2
\] 
On the other hand, using $|a-b| \leq |\max(0, a) - \max(0,a)|$ and the fact that max pool layer can be written as a convolutional layer such that only one weight is 1 and others are 0. We can further state that for ReLU and max-pool layers,
\[
\|\mathbf{x}^d - \mathbf{\tilde{x}}^d\|_2 \leq  \|\mathbf{x}^{d-1} - \mathbf{\tilde{x}}^{d-1}\|_2
\] 

Combining with the Lipschitz constant of soft-max layer,
\[
\|CNN(\mathbf{x};\mathbf{w}) - CNN(\mathbf{\tilde{x}};\mathbf{w})\|_2 \leq   \frac{\sqrt{C-1}}{C} \alpha^{n_c+n_{fc}}  \|\mathbf{x}-\mathbf{\tilde{x}}\|_2
\]
Since the loss function is $l_2$ as well, this concludes the proof.
\end{proof}

\section{Proof for Theorem 1}
In order to prove the Theorem 1, we extend the robustness bound from \cite{robust}.
\begin{proof}
\begin{small}
We will start with
\[
\begin{aligned}
&\left|E_{\mathbf{x},y \sim p_\mathcal{Z}}[l(\mathbf{x},y, A_\mathbf{s})] - \frac{1}{n}\sum_{i \in [n]} l(\mathbf{x}_i,y_i,A_\mathbf{s})\right| \\
&\overset{(a)}{\leq} \left|\sum_{j \in [K]} E[l(\mathbf{x},y)| (\mathbf{x},y) \in C_j] \mu_{j} -  \sum_{j \in [K]} E[l(\mathbf{x},y)| (\mathbf{x},y) \in C_j] \frac{|n_j|}{n} \right| \\
 &+  \left| \sum_{j \in [K]} E[l(\mathbf{x},y)| (\mathbf{x},y) \in C_j] \frac{|n_j|}{n}  - \frac{1}{n}\sum_{i \in [n]} l(\mathbf{x}_i,y_i) \right| \\
  &\overset{(b)}{\leq}\left|\sum_{j \in [K]} E[l(\mathbf{x},y)| (\mathbf{x},y) \in C_j] (\mu_{j} -   \frac{|n_j|}{n}) \right|
 +\frac{1}{n} \left|\sum_{j \in [K]} \sum_{i \in n_j} E[l(\mathbf{x},y)| (\mathbf{x},y) \in C_j]  - l(\mathbf{x}_i,y_i)\right| \\
   &\overset{(c)}{\leq} \left|\sum_{j \in [K]} E[l(\mathbf{x},y)|z \in C_j] (\mu_{j} -   \frac{|n_j|}{n})\right| + \lambda^l \epsilon  \\
 \end{aligned}
\]
\end{small}

Here, with brevity we denoted $l(\mathbf{x},y, A_\mathbf{s})$ as $l(\mathbf{x},y)$. In $(a)$, we used the fact that the space has an $\epsilon$ cover; and denote the cover as $\{C_j\}_{j \in [K]}$ such that each $C_j$ has diameter at most $\epsilon$. We further defined an auxiliary variable $\mu_j=p((\mathbf{x},y) \in C_j)$ and $n_j = \sum_i \mathds{1}[(\mathbf{x}_i,y_i) \in C_j]$ and used triangle inequality. In $(b)$, we used $i \in n_j$ to represent $(\mathbf{x}_i,y_i) \in C_j$. Finally, in $(c)$ we used the fact that each ball has diameter at most $\epsilon$ and the loss function is $\lambda^l$-Lipschitz. 
%Now, we will use the zero-loss of the classifier with Lipschitz continuity as;
%\[
%\begin{aligned}
%\left|\frac{1}{n}\sum_i l(A_s,x_i) \right| &= \left|\frac{1}{n}\sum_{j \notin {s(i)}_{i\in [M]}} l(A_s,x_i)  \right| \\
%&\leq  \frac{1}{n}\sum_{j \notin {s(i)}_{i\in [M]}} \lambda  \left| \mathbf{x}_i - \mathbf{x}_k\right | \leq \frac{n-m}{n} \lambda \tilde{\epsilon}
%\end{aligned}
%\]
%Combining both,
%\[
%\begin{aligned}
%&E[l(A_s,z)] \leq  \left|E[l(A_s,z)] - \frac{1}{n}\sum_i l(A_s,x_i) \right|  \\ &+ \left|\frac{1}{n}\sum_i l(A_s,x_i) - \frac{1}{m}\sum_i l(A_s,x_{s(i)}) \right| \\
%&\leq \left|\sum_{j} E[l(A_s,z)|z \in C_j] (\mu_{j} -   \frac{|N_j|}{n})\right| +\epsilon(s) + \frac{n-m}{n} \lambda \tilde{\epsilon}
%\end{aligned}
%\]

We can bound $E[l(\mathbf{x},y)|z \in C_j]$ with maximum loss $L$ and use Breteganolle-Huber-Carol inequality (\emph{cf} Proposition A6.6 of \cite{wellner}) in order to bound $\sum_{j} \mu_{j} -   \frac{|n_j|}{n}$. 

Combining all; with probability at least $(1-\delta)$,
\[
\left|E_{\mathbf{x},y \sim p_\mathcal{Z}}[l(\mathbf{x},y, A_\mathbf{s})] - \frac{1}{n}\sum_{i \in [n]} l(\mathbf{x}_i,y_i,A_\mathbf{s})\right| \leq  \lambda^l \epsilon + L \sqrt{\frac{2K\log 2 + 2\log (1/\delta)}{n}}
\]
\end{proof}

\section{Proof for Theorem 2}
Before starting our proof, we state the Claim 1 from \cite{BerlindU15}. Fix some $p,p^\prime \in [0,1]$ and $y^\prime \in \{0,1\}$. Then,
\[
p_{y \sim p}(y \neq y^\prime) \leq p_{y \sim p^\prime}(y \neq y^\prime) + |p - p^\prime|
\]
\begin{proof}
We will start our proof with bounding $E[l(\mathbf{x}_i,y_i)]$. We have a condition which states that there exists and $\mathbf{x}_j$ in $\delta$ ball around $\mathbf{x}_i$ such that $\mathbf{x}_j$ has $0$ loss.
\[
\begin{aligned}
E[l(\mathbf{x}_i,y_i)] &= \sum_{k\in [C]} p_{y_i \sim \eta_k(\mathbf{x}_i)}(y_i = k) l(\mathbf{x}_i,k) \\
&\overset{(d)}{\leq} \sum_{k\in [C]} p_{y_i \sim \eta_k(\mathbf{x}_j)}(y_i = k) l(\mathbf{x}_i,k) \\ &\quad+ \sum_{k\in [C]}  |\eta_k(\mathbf{x}_i)-\eta_k(\mathbf{x}_j)| l(\mathbf{x}_i,k) \\
&\overset{(e)}{\leq} \sum_{k\in [C]} p_{y_i \sim \eta_k(\mathbf{x}_j)} (y_i = k) l(\mathbf{x}_i,k) + \delta \lambda^\eta L C\\ 
\end{aligned}
\]
With abuse of notation, we represent \mbox{$\{y_i=k\} \sim \eta_k(\mathbf{x}_i)$} with \mbox{$y_i \sim \eta_k(\mathbf{x}_i)$}. We use Claim 1 in $(d)$, and Lipschitz property of regression function and bound of loss in $(d)$. Then, we can further bound the remaining term as; 
\[
\begin{aligned}
&\sum_{k\in [C]} p_{y_i \sim \eta_k(\mathbf{x}_j)} (y_i = k) l(\mathbf{x}_i,k) \\
&= \sum_{k\in [C]} p_{y_i \sim \eta_k(\mathbf{x}_j)} (y_i = k) [l(\mathbf{x}_i,k) - l(\mathbf{x}_j,k) ] \\ &\quad+ \sum_{k\in [C]} p_{y_i \sim \eta_k(\mathbf{x}_j)} (y_i = k) l(\mathbf{x}_j,k) \\
&\leq \delta \lambda^l
\end{aligned}
\]
where last step is coming from the fact that the trained classifier assumed to have $0$ loss over training points. If we combine them,
\[
E[l(\mathbf{x_i},y_i)] \leq \delta( \lambda^l+\lambda^\mu LC)
\]
We further use the Hoeffding's Bound and conclude that with probability at least $1 - \gamma$,
\[
\frac{1}{n}\sum_{i \in [n]} l(\mathbf{x}_i,y_i) \leq \delta (\lambda^l + \lambda^\mu LC)+ 
\sqrt{\frac{L \log(1/\gamma)}{2n}}
%\frac{\log(1/(1-\delta)){2n}
\]
\end{proof}

\section{Unsupervised subset selection} 

Since our algorithm does not use any uncertainty information, we can also use it for unsupervised subset selection. In other words, given an informative distance function, we can use our algorithm to choose subset of the training set. In unsupervised subset selection problem, the constraint is using subset of the dataset due to the resource constraints. Given a dataset size constraint, the problem is choosing a subset of examples such that the trained model will perform as closely as possible to the model trained on entire dataset. One big difference to active learning is lack of iteration. Given a budget, subset is selected and not refined/improved.


We perform this experiment with $l_2$ distance of features learned with no supervision as $\Delta(\cdot,\cdot)$. We use \cite{improved_gan} for unsupervised feature learning using their shared source code. We plot the accuracy vs dataset size using both our algorithm and uniformly random selection in Figure~\ref{fig:scat} and conclude that our algorithm is effective for the problem of unsupervised subset selection.  

\begin{figure}[t]
\includegraphics[width=\textwidth]{u_100_2.pdf}
\caption{Experiment on unsupervised subset selection. We use our algorithm with ImprovedGAN\cite{improved_gan} features in unsupervised subset selection setup. Results suggest that our algorithm is more effective than uniform random sampling .}
\label{fig:scat}
\end{figure}

%{\small
\bibliography{active_adversarial} 
\bibliographystyle{abbrv}
%}




\end{document}
